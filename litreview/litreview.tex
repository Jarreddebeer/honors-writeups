\documentclass[prodmode,acmtecs]{acmsmall}
\usepackage[ruled]{algorithm2e}

\acmYear{2015}
\acmMonth{4}

\begin{document}
\title{The Performance Characteristics of Astronomical Source Finders}
\author{YASEEN HAMDULAY
\affil {University of Cape Town}
}

\begin{abstract}
MeerKAT and ASKAP will run the biggest Hydrogen surveys ever completed. Our current source
finders will be unable to cope with such large surveys. We will look at different source
finding algorithms and techniques for accelerating them.
\end{abstract}
\keywords{Source Finding, Astronomy, GPU}

\acmformat{Yaseen Hamdulay, 2015. The Performance Characteristics of Astronomical Source Finders}

\maketitle

\section{Introduction}
Discuss MeerKAT, ASKAP and the performance issues expected with these larger Hydrogen surveys.
\cite{holwerda2010trumpeting}
\cite{whiting2012source}
\cite{floer2014source}

\section{Comparison of Existing Source Finding Algorithms}
 \cite{westerlund2012assessing}
 \cite{popping2012comparison}

    \subsection{CNHI}
Characterises noise with no parameterization.
    \cite{jurek2012characterised}

    \subsection{2D-1D Wavelet Reconstruction}
    \cite{floer20122d}

    \subsection{Smooth and Clip Filter}
\cite{serra2012atlas3d}

    \subsection{Parallel Gaussian}
This algorithm is already GPU'ified. Discuss it's mechanism and performance characteristics.
        \cite{westerlund2014framework}

\section{Source Finding Platforms}
    \subsection{Duchamp}
Most well-known source finder. Makes use of a thresholding algorithm
        \cite{whiting2012duchamp}
    \subsection{SoFiA}
Newer source-finder written mostly in Python that implements most
existing source-finding algorithms.
        \cite{serra2015sofia}

\section{Acceleration Techniques}
    \subsection{Multi-core CPU}
Discuss current attempts to make use of parallelism on the CPU to improve performance
and it's limitations.
        \cite{westerlund2014framework}
        \cite{scott}

    "Acceleration of the noise suppression component of the DUCHAMP source-finder."
    \subsection{GPU}
Talk about already GPU'ified source-finding algorithms and the fact that athis needs
to be repeated for all source-finding algorithms (this is our niche).
    \cite{laidler2013detection}
    \cite{hassan2011unleashing}

\section{Existing Accelerated Source Finders}
Discuss \cite{westerlund2015performance} and the algorithm they use.

\section{Conclusions}
Acceleration of source finding algorithms is necessary before ASKAP and MeerKAT becomes
available. We hope that GPU'ification of these algorithms will allow us to handle the
data coming from the new telescopes.

\bibliographystyle{ACM-Reference-Format-Journals}
\bibliography{structure}

\end{document}
